\section{Ações de Combate}
\subsection{Corpo a Corpo}

\noindent {\large DESEMBAINHAR} --- \textit{Rápida}. Não há rolagem. Você também pode usar essa ação para pegar uma arma ou item AO ALCANCE DAS MÃOS (MdJ, pg. 93).\\

\noindent {\large DERRUBAR} --- \textit{Rápida}. Role LUTA. Pode usar escudo ou arma com a característica GANCHO. Caso o oponente tenha Força maior que a sua, são necessários dois \epsdice{6} para obter sucesso. Caso tenha sucesso, o oponente fica CAÍDO. Pode ser ESQUIVADO ou BLOQUEADO com um escudo (MdJ, pg. 93).\\

\noindent {\large DESARMAR} --- \textit{Rápida}. Rule LUTA. Pode usar uma arma. Precisa de um \epsdice{6} para desarmar uma arma de uma mão. Uma arma de duas mãos requer dois. Caso tenha sucesso, a arma do oponente cai AO ALCANCE DAS MÃOS. Você não pode DESARMAR um escudo. Pode ser ESQUIVADO ou BLOQUEADO (MdJ, pg. 93).\\

\noindent {\large FINTAR} --- \textit{Rápida}. Não há rolagem. Troque de iniciativa com um oponente ao seu alcance a partir do próximo turno (MdJ, pg. 93).\\

\noindent {\large ARREMETER} --- \textit{Rápida}. Não há rolagem. Feito antes de atacar com uma arma PESADA. Se o ataque
tiver sucesso, o dano causado é aumentado em +1 (MdJ, pg. 93).

\begin{hpbox}{4}
    \begin{center}
        \quad\vspace{10pt}
        
        {\large\textbf{SEQUÊNCIA DE COMBATE}}\\
        \quad\vspace{10pt}

        COMPRE A INICIATIVA\\
        \textit{Compre uma carta (1-10)}\\
        ------\\
        DECLARE A AÇÃO\\
        \textit{2 rápidas ou 1 lenta + 1 rápida}\\
        $\downarrow$\\
        ROLE A AÇÃO\\
        \textit{Ataques: para cada \epsdice{6} extra, +1 dano}.\\
        $\downarrow$\\
        \begin{tcolorbox}[colback=white,colframe=black,arc=0pt,halign=center]
        DECLARE A REAÇÃO\\
        \textit{ESQUIVAR ou BLOQUEAR}\\
        $\downarrow$\\
        ROLE A REAÇÃO\\
        \textit{Para cada \epsdice{6}, -\epsdice{6} do ataque}.\\
        $\downarrow$\\
        ROLE A ARMADURA\\
        \textit{Para cada \epsdice{6}, -1 dano}.\\
        $\downarrow$\\
        APLIQUE O DANO\\
        \end{tcolorbox}
        $\downarrow$\\
        PRÓXIMO NA INICIATIVA\\
    \end{center}
\end{hpbox}

\noindent {\large GOLPEAR} --- \textit{Lento}. Role LUTA com uma arma com a característica CORTE ou ESMAGAMENTO. Pode ser ESQUIVADO (+2) ou BLOQUEADO. Para cada \epsdice{6} extra, +1 de dano (MdJ, pg. 92).\\

\noindent {\large ESTOCAR} --- \textit{Lento}. Role LUTA com uma arma com a característica PONTIAGUDA. Pode ser ESQUIVADO ou BLOQUEADO (-2 com arma, +2 com escudo). Para cada \epsdice{6} extra, +1 de dano (MdJ, pg. 92).\\

\noindent {\large SOCO/CHUTE/MORDIDA} --- \textit{Lento}. Role LUTA. Pode ser ESQUIVADO ou BLOQUEADO (+2). Para cada \epsdice{6} extra, +1 de dano (MdJ, pg. 92).\\

\noindent {\large AGARRAR} --- \textit{Lento}. Rule LUTA sem armas. Pode ser ESQUIVADO ou BLOQUEADO. Caso tenha sucesso, você e o oponente ficam CAÍDOS. O oponente larga as armas e não pode se mover; pode apenas LIBERTAR-SE (ação lenta, teste de LUTA contestado).

Enquanto estiver agarrando, a única ação que você pode fazer é o ATAQUE ATRACADO. Funciona como SOCO/CHUTE/MORDIDA, mas é uma ação \textit{Rápida} e não pode ser ESQUIVADA ou BLOQUEADA. (MdJ, pg. 92).\\

\begin{figure}[h]
    \centering
    \includegraphics[width=\linewidth]{images/art/warrior.png}
\end{figure}

\subsection{À Distância}

\begin{hpbox}{3}
     \quad\vspace{2pt}
     
    \begin{center}
        {\large\textbf{Alcance do Disparo}}
    \end{center} 
    \vspace{5pt}
    
    \rowcolors{2}{white}{rowgray}
    \begin{adjustbox}{width=\linewidth}
    \begin{tabular}{lc}
        \thead{ALCANCE} & \thead{MODIFICADOR} \\\toprule
        Às Mãos & -3/+3\\
        Perto & ---\\
        Curta & -1\\
        Longa & -2\\
        Distante & -3 (Mira)\\
    \end{tabular}
    \end{adjustbox}
\end{hpbox}

\noindent {\large PREPARAR} --- \textit{Rápida}. Não há rolagem. Ação necessária antes de ATIRAR. Quando PREPARADO, você pode apenas ATIRAR ou MIRAR. Se tomar qualquer outra ação, precisará PREPARAR novamente. Bestas, por sua vez, são CARREGADAS (ação \textit{Lenta}) (MdJ, pg. 97).\\

\noindent {\large MIRAR} --- \textit{Rápida}. Não há rolagem. Feito antes de ATIRAR. Adicione +1 ao ataque. Você não pode MIRAR em um alvo atento AO ALCANCE DAS MÃOS (MdJ, pg. 98).\\

\noindent {\large ATIRAR} --- \textit{Lenta}. Precisa, antes, PREPARAR. Role PONTARIA com uma arma à distância. Pode ser ESQUIVADO ou BLOQUEADO com um escudo ou arma com característica APARAR e pelo menos a categoria 2 no talento relevante para o tipo de arma. Para cada \epsdice{6} extra, +1 de dano (MdJ, pg. 98).\\\quad\\\quad\\\quad\\

\subsection{Defesa e Movimento}

\noindent {\large ESQUIVAR} --- \textit{Rápida (Reação)}. Role MOVIMENTAÇÃO. Quando você ESQUIVA, você fica CAÍDA. Você pode escolher continuar em pé, mas apenas ao custo de uma penalidade de -2. Para cada \epsdice{6} seu, cancele um \epsdice{6} do oponente (MdJ, pg. 92; 98).\\

\noindent {\large BLOQUEAR} --- \textit{Rápida (Reação)}. Role LUTA. Pode usar uma arma ou um escudo. Se você BLOQUEIA com uma arma sem a característica APARAR, você recebe uma penalidade de -2. Para cada \epsdice{6} seu, cancele um \epsdice{6} do oponente (MdJ, pg. 92; 98).\\

\noindent {\large LEVANTAR} --- \textit{Rápida}. Não há rolagem. Levante ou fique CAÍDA. Ataques corpo a corpo contra oponentes caídos recebem um bônus de +2. (MdJ, pg. 93).\\

\noindent {\large CORRER} --- \textit{Rápida}. Não há rolagem, a não ser que esteja entrado em uma zona ACIDENTADA (MOVIMENTAÇÃO). Mova para uma zona adjacente, ou saia da distância PERTO para AO ALCANCE DAS MÃOS. Se CAÍDA, RASTEJE (ação \textit{Lenta}) (MdJ, pg. 93).\\

\noindent {\large RECUAR} --- \textit{Rápida}. Role MOVIMENTAÇÃO. Usado ao invés de CORRER, caso esteja AO ALCANCE DAS MÃOS. Mova para a distância PERTO. Se você fracassar, você se move, mas o seu inimigo ganha um ataque livre contra você que não pode ser ESQUIVADO ou BLOQUEADO (MdJ, pg. 93).\\

\noindent {\large FUGIR} --- \textit{Lento}. Role MOVIMENTAÇÃO. Caso tenha sucesso, você escapa do combate. Não pode FUGIR através de um adversário ou caso tenha um oponente AO ALCANCE DAS MÃOS. Você precisa FUGIR na direção de onde veio (MdJ, pg. 93).

\begin{figure}
    \centering
    \includegraphics[width=\linewidth]{images/art/execute.png}
\end{figure}

\begin{figure*}
    \begin{hpbox*}
        \begin{center}
                \rowcolors{2}{rowgray}{white}
                \begin{adjustbox}{width=\linewidth}
                \begin{tabular}{lcccc}
                    \multicolumn{5}{c}{\large\textbf{Corpo a Corpo}}\\\addlinespace
                    \rowcolor{white}\thead{Ação} & \thead{Tipo} & \thead{Requisito} & \thead{Perícia} & \thead{Página} \\\toprule
                    Desembainhar & Rápida & -- & -- & 93 \\
                    Derrubar & Rápida & -- & Luta & 93 \\
                    Desarmar & R\'apida & Seu alvo est\'a segurando uma arma & Luta & 93 \\
                    Fintar & R\'apida & Inimigo ÀS MÃOS & -- & 93 \\
                    Arremeter & R\'apida & Executar antes de um ataque PESADO & -- & 93 \\
                    Golpear & Lenta & Arma de CORTE ou de CONTUS\~AO & Luta & 92 \\
                    Estocar & Lenta & Arma PERFURANTE & Luta & 92 \\
                    Soco/Chute/Mordida & Lenta & Desarmado & Luta & 92 \\
                    Agarrar & Lenta & Desarmado & Luta & 92 \\
                    \addlinespace\addlinespace\multicolumn{5}{c}{\large\textbf{À Distância}}\\\addlinespace\addlinespace
                    Preparar & R\'apida & Arma de longo alcance & -- & 97 \\
                    Mirar & R\'apida & Arma de longo alcance, PERTO + & -- & 98 \\
                    Atirar & Lenta & Voc\^e preparou a arma & Pontaria & 98 \\\addlinespace\addlinespace
                    \multicolumn{5}{c}{\large\textbf{Defesa e Movimento}}\\\addlinespace\addlinespace
                    Esquivar & Rápida & -- & Movimento & 92; 98 \\
                    Bloquear & Rápida & Escudo ou arma & Luta & 92; 98 \\
                    Levantar & Rápida & Você está CAÍDO & -- & 93 \\
                    Correr & Rápida & Nenhum inimigo ÀS MÃOS & ---/Movimento & 88--89 \\
                    Recuar & Rápida & Inimigo ÀS MÃOS & Movimento & 93 \\
                    Fugir & Lenta & Nenhum inimigo ÀS MÃOS & Movimento & 89 \\\addlinespace\addlinespace
                    \rowcolor{white}\multicolumn{5}{c}{\large\textbf{Outros}}\\\addlinespace\addlinespace
                    \rowcolor{rowgray}Usar item & Rápida & Varia & Varia & -- \\
                    \rowcolor{white}Palavra de poder & Rápida & Você é um Druida ou Feiticeiro & -- & 116--142 \\
                    \rowcolor{rowgray}Conjurar magia & Lenta & Você é um Druida ou Feiticeiro & -- & 116--142 \\
                    \rowcolor{white}Curar outro & Lenta & Nenhum inimigo ÀS MÃOS & Cura/Atuação & 56 \\
                    \rowcolor{rowgray}Persuadir & Lenta & O oponente pode ouvir você & Manipulação & 55 \\
                    \rowcolor{white}Provocar & Lenta & O oponente pode ouvir você & Atuação & 57 
                \end{tabular}
            \end{adjustbox}
        \end{center}
    \end{hpbox*}
\end{figure*}