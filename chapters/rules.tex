\section{Testes}

\begin{hpbox}{1}
    
    \begin{center}
        {\large\textbf{A REGRA DE OURO}}
    \end{center}
    Para realizar uma ação, role D6s iguais ao seu:

    \begin{center}
        \textbf{Atributo + Perícia + Equipamento}
    \end{center}

    Para ter sucesso, você deve conseguir ao menos um {\normalsize\epsdice{6}}
\end{hpbox}

\subsection{Modificadores}
Quando você possui um MODIFICADOR, adicione ou remova Dados de Perícia da sua rolagem. Se você tiver Dados de Perícia negativos, \epsdice{6} nos Dados de Perícia cancelam demais \epsdice{6} da rolagem.

\begin{figure}[h]
    \centering
    % \includegraphics[width=.69\linewidth]{images/art/thinker.png}
    % \includegraphics[width=.88\linewidth]{images/art/liar.png}
    \includegraphics[width=.82\linewidth]{images/art/dice.png}
\end{figure}

\subsection{Forçando a Rolagem (Esforço)}
Uma vez por ação, caso obter um \epsdice{6} altere o resultado, você pode FORÇAR A ROLAGEM. Role novamente todos \epsdice{2} a \epsdice{5} (assim como \epsdice{1} para os Dados de Perícia). 

Caso obtenha um \epsdice{6}, você obtém um sucesso, conforme a Regra de Ouro. Caso role \epsdice{1}s, você obtém \textit{ruína} (isso ativa os \epsdice{1} da primeira rolagem). O efeito da \textit{ruína} varia por tipo de dado:
\begin{itemize}
    \item Dados Base: Sofra 1 ponto de dano no atributo relevante e ganhe 1 Ponto de Força de Vontade (PFV). 
    \item Dados de Equipamento: O bônus do seu equipamento é reduzido em 1.
    \item Dado de Perícia: Nada acontece.
\end{itemize}

\section{Força de Vontade}
Quando você FORÇA A ROLAGEM e rola uma \textit{ruína} em seus Dado Base, você ganha PFV (max. 10). Esses pontos são usados para ativar seus Talentos de Ascendência e Profissão, asism como conjurar Magias

\section{Golpe de Misericórdia}
Para matar uma criatura inteligente incapacitada (Força ou Agilidade ACABADA), role Empatia. Caso você \textit{falhe}, você a mata; gaste 1 PfV e sofra 1 de dano em Empatia. Caso você tenha sucesso, você não consegue realizar a ação.

\pagebreak\section{Acabado}

\begin{figure*}[b]
    \centering
    \includegraphics[width=\linewidth]{images/art/help.png}
\end{figure*}

Quando um atributo cai a 0, você está ACABADO. Caso sua Força ou Astúcia fique ACABADA e isso não tenha ocorrido por causa de um ESFORÇO, você sobre um Ferimento Grave (MdJ, pg. 196-200).

Você pode se recuperar de estar ACABADO de três maneiras:
\begin{itemize}
    \item Um aliado rola CURA ou ATUAÇÃO; recupere 1 ponto por cada \epsdice{6}.
    \item Você espera D6 horas e recupera 1 ponto.
    \item Você DESCANSA ou DORME um Quarto de Dia e recupera todos os pontos
\end{itemize}

\section{Morte}
Caso você sofra um Ferimento Grave LETAL, alguém deve conseguir um sucesso em um teste de CURA para salvar você.

Caso recupere pontos de atributo, você pode tentar você mesmo (penalidade -2). Uma personagem pode tentar lhe \mbox{CURAR} apenas uma vez. Se você não for \mbox{CURADO} no limite de tempo especificado pelo seu Ferimento Grave, você morre.

\clearpage\section{Jornada}
O dia é dividido em Quartos de Dia, nomeadamente, \textit{Manhã, Dia, Anoitecer e Noite}. Você pode realizar 1 Atividade a cada Quarto de Dia.

\subsection{Atividades}
{\large CAMINHAR} --- \textit{Obrigatória (todos)}. Vocês se deslocam 2 hexágonos por Quarto de Dia quando CAMINHAM em um TERRENO ABERTO (metade em TERRENO DIFICULTOSO). Caso desejem  CAMINHAR por mais que 2 Quartos de Dia, é necessária uma MARCHA FORÇADA. Rolem RESILIÊNCIA. Os falhantes sofrem 1 de dano em Agilidade e precisam DESCANSAR ou DORMIR.\\

\noindent {\large DESBRAVAR} ---  \textit{Obrigratória}. Quando você DESBRAVA, você é o \textit{desbravador}. Ao entrar em um novo hexágono, role SOBREVIVÊNCIA. Escuridão complica essa rolagem (MdJ, pg. 147). Caso falhem, vocês entram no hexágono, mas sofrem um infortúnio (MdJ, pg. 148-149).\\

\noindent {\large MANTER VIGÍLIA} --- \textit{Recomendado}. Quando você MANTÉM VIGÍLIA, você é a \textit{vigia}. Role PATRULHAR uma vez por Quarto de Dia. Se você falhar, você não observa o perigo aproximando-se.\\

\noindent {\large COLETAR} --- Quando vocês param para COLETAR, role SOBREVIVÊNCIA, modificando pelo TERRENO e ESTAÇÃO (MdJ, pg.145;151). Caso tenha sucesso, você entra 1 unidade de vegetais e água por cada \epsdice{6}. Caso falhe, sofre um infortúnio (MdJ, pg. 150).\\

\noindent {\large CAÇAR} --- Quando você para para CAÇAR, você rola SOBREVIVÊNCIA para encontrar sua presa. Role PONTARIA (arma) ou SOBREVIVÊNCIA (armadilha) para matá-la. Essa rolagem é modificada pela dificuldade do animal (MdJ, pg. 152). Caso tenha sucesso, você recupera CARNE e PELES. Caso falhe, sofre um infortúnio (MdJ, pg. 150). \\

\noindent {\large PESCAR} --- Quando você para para PESCAR, role SOBREVIVÊNCIA + Equipamento. Caso tenha sucesso, você coleta 1 PEIXE por cada \epsdice{6}. Caso falhe, sofre um infortúnio (MdJ, pg. 153).\\

\noindent {\large FAZER ACAMPAMENTO} --- Quando você para para FAZER ACAMPAMENTO, role SOBREVIVÊNCIA. Caso falhe, você sofre um infortúnio (MdJ, pg. 154-155).

Vocês podem optar por não montar acampamento e dormir no chão. Nesse caso, cada personagem rola SOBREVIVÊNCIA para encontrar um local para DORMIR. Falhar implica não DORMIR e, portanto, ficar INSONE, assim como sofrer efeitos de FRIO.\\

\noindent {\large DESCANSAR/DORMIR} --- Quando você DESCANSA e você não está sofrendo de uma CONDIÇÃO que impeça RECUPERAÇÃO, você recupera todos os pontos de atributo perdidos. Se você for interrompido por uma atividade intensa (como um combate), você não DESCANSA. DORMIR é como DESCANSAR, mas você fica INSONE caso passe mais de 1 dia completo sem DORMIR.  

\begin{figure}[b]
    \centering
    \includegraphics[width=0.85\linewidth]{images/art/journey.png}
\end{figure}

\begin{figure*}[t]
\begin{hpbox*}    
\centering
\rowcolors{2}{white}{rowgray}
{\large\textbf{Condições e Consequências}}
\begin{adjustbox}{width=\linewidth}
\begin{tabular}{lcccc}
    \thead{Nome} & \thead{Dano} & \thead{Restrição} & \thead{Recuperação} & \thead{Página} \\\toprule
    Faminta  & 1 Força/semana & Não recupera Força & 1 Comida & 111 \\
    Desidratada & 1 Agilidade e Força/dia & Não recupera Atributos & 1 Água  & 111 \\
    Insone & 1 Astúcia/dia & Não recupera Astúcia & Dormir  & 111 \\
    Hipotérmica & 1 Astúcia e Força/teste & Não recupera Astúcia e Força & Aquecer & 111 
\end{tabular}
\end{adjustbox}
\end{hpbox*}
\end{figure*}

\section{Dados de Recurso}
Seu estoque de COMIDA, ÁGUA, FLECHAS e TOCHAS é contabilizado através de Dados de Recurso (D6 a D12). Cada vez que usar um desses itens, role o respectivo dado. Em um \epsdice{1} ou \epsdice{2}, diminua seu Dado de Recurso em 1 passo (D10 para D8, por exemplo). Quando rolar \epsdice{1} ou \epsdice{2} em um D6, seu estoque acabou.

% \begin{figure}[h]
%     \centering
%     \includegraphics[width=\linewidth]{images/art/items.png}
% \end{figure}

\section{Sobrecarga}
Você pode temporariamente carregar mais que o seu limite normal de carga (Força$\times$2). Quando o fazendo, você precisa rolar RESILIÊNCIA quando quiser CORRER em combate ou CAMINHAR por um Quarto de Dia. Caso falhe, você precisa largar o que estiver carregando, parar onde estiver ou sofrer 1 ponto de dano em Agilidade e seguir em frente.

\section{Experiência}
Você adquire Ponto de Experiência (XP) ao realizar ações aventurescas (MdJ, pg. 39). Quando você DESCANSA ou DORME, pode gastar XP para melhorar ou aprender Perícias e Talentos. Gaste XP igual ao Nível de Perícia desejado multiplicado por 5; para Talento, por 3. Isso não vale para Talentos Mágicos (MdJ, Cap. 6)

\begin{figure}[h]
    \centering
    % \includegraphics[width=\linewidth]{images/art/climb.png}
    \includegraphics[width=.9\linewidth]{images/art/horse.png}
\end{figure}
